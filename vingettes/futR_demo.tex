% Options for packages loaded elsewhere
\PassOptionsToPackage{unicode}{hyperref}
\PassOptionsToPackage{hyphens}{url}
%
\documentclass[
]{article}
\usepackage{amsmath,amssymb}
\usepackage{lmodern}
\usepackage{ifxetex,ifluatex}
\ifnum 0\ifxetex 1\fi\ifluatex 1\fi=0 % if pdftex
  \usepackage[T1]{fontenc}
  \usepackage[utf8]{inputenc}
  \usepackage{textcomp} % provide euro and other symbols
\else % if luatex or xetex
  \usepackage{unicode-math}
  \defaultfontfeatures{Scale=MatchLowercase}
  \defaultfontfeatures[\rmfamily]{Ligatures=TeX,Scale=1}
\fi
% Use upquote if available, for straight quotes in verbatim environments
\IfFileExists{upquote.sty}{\usepackage{upquote}}{}
\IfFileExists{microtype.sty}{% use microtype if available
  \usepackage[]{microtype}
  \UseMicrotypeSet[protrusion]{basicmath} % disable protrusion for tt fonts
}{}
\makeatletter
\@ifundefined{KOMAClassName}{% if non-KOMA class
  \IfFileExists{parskip.sty}{%
    \usepackage{parskip}
  }{% else
    \setlength{\parindent}{0pt}
    \setlength{\parskip}{6pt plus 2pt minus 1pt}}
}{% if KOMA class
  \KOMAoptions{parskip=half}}
\makeatother
\usepackage{xcolor}
\IfFileExists{xurl.sty}{\usepackage{xurl}}{} % add URL line breaks if available
\IfFileExists{bookmark.sty}{\usepackage{bookmark}}{\usepackage{hyperref}}
\hypersetup{
  pdftitle={futR},
  pdfauthor={Kirstin Holsman},
  hidelinks,
  pdfcreator={LaTeX via pandoc}}
\urlstyle{same} % disable monospaced font for URLs
\usepackage[margin=1in]{geometry}
\usepackage{color}
\usepackage{fancyvrb}
\newcommand{\VerbBar}{|}
\newcommand{\VERB}{\Verb[commandchars=\\\{\}]}
\DefineVerbatimEnvironment{Highlighting}{Verbatim}{commandchars=\\\{\}}
% Add ',fontsize=\small' for more characters per line
\usepackage{framed}
\definecolor{shadecolor}{RGB}{248,248,248}
\newenvironment{Shaded}{\begin{snugshade}}{\end{snugshade}}
\newcommand{\AlertTok}[1]{\textcolor[rgb]{0.94,0.16,0.16}{#1}}
\newcommand{\AnnotationTok}[1]{\textcolor[rgb]{0.56,0.35,0.01}{\textbf{\textit{#1}}}}
\newcommand{\AttributeTok}[1]{\textcolor[rgb]{0.77,0.63,0.00}{#1}}
\newcommand{\BaseNTok}[1]{\textcolor[rgb]{0.00,0.00,0.81}{#1}}
\newcommand{\BuiltInTok}[1]{#1}
\newcommand{\CharTok}[1]{\textcolor[rgb]{0.31,0.60,0.02}{#1}}
\newcommand{\CommentTok}[1]{\textcolor[rgb]{0.56,0.35,0.01}{\textit{#1}}}
\newcommand{\CommentVarTok}[1]{\textcolor[rgb]{0.56,0.35,0.01}{\textbf{\textit{#1}}}}
\newcommand{\ConstantTok}[1]{\textcolor[rgb]{0.00,0.00,0.00}{#1}}
\newcommand{\ControlFlowTok}[1]{\textcolor[rgb]{0.13,0.29,0.53}{\textbf{#1}}}
\newcommand{\DataTypeTok}[1]{\textcolor[rgb]{0.13,0.29,0.53}{#1}}
\newcommand{\DecValTok}[1]{\textcolor[rgb]{0.00,0.00,0.81}{#1}}
\newcommand{\DocumentationTok}[1]{\textcolor[rgb]{0.56,0.35,0.01}{\textbf{\textit{#1}}}}
\newcommand{\ErrorTok}[1]{\textcolor[rgb]{0.64,0.00,0.00}{\textbf{#1}}}
\newcommand{\ExtensionTok}[1]{#1}
\newcommand{\FloatTok}[1]{\textcolor[rgb]{0.00,0.00,0.81}{#1}}
\newcommand{\FunctionTok}[1]{\textcolor[rgb]{0.00,0.00,0.00}{#1}}
\newcommand{\ImportTok}[1]{#1}
\newcommand{\InformationTok}[1]{\textcolor[rgb]{0.56,0.35,0.01}{\textbf{\textit{#1}}}}
\newcommand{\KeywordTok}[1]{\textcolor[rgb]{0.13,0.29,0.53}{\textbf{#1}}}
\newcommand{\NormalTok}[1]{#1}
\newcommand{\OperatorTok}[1]{\textcolor[rgb]{0.81,0.36,0.00}{\textbf{#1}}}
\newcommand{\OtherTok}[1]{\textcolor[rgb]{0.56,0.35,0.01}{#1}}
\newcommand{\PreprocessorTok}[1]{\textcolor[rgb]{0.56,0.35,0.01}{\textit{#1}}}
\newcommand{\RegionMarkerTok}[1]{#1}
\newcommand{\SpecialCharTok}[1]{\textcolor[rgb]{0.00,0.00,0.00}{#1}}
\newcommand{\SpecialStringTok}[1]{\textcolor[rgb]{0.31,0.60,0.02}{#1}}
\newcommand{\StringTok}[1]{\textcolor[rgb]{0.31,0.60,0.02}{#1}}
\newcommand{\VariableTok}[1]{\textcolor[rgb]{0.00,0.00,0.00}{#1}}
\newcommand{\VerbatimStringTok}[1]{\textcolor[rgb]{0.31,0.60,0.02}{#1}}
\newcommand{\WarningTok}[1]{\textcolor[rgb]{0.56,0.35,0.01}{\textbf{\textit{#1}}}}
\usepackage{graphicx}
\makeatletter
\def\maxwidth{\ifdim\Gin@nat@width>\linewidth\linewidth\else\Gin@nat@width\fi}
\def\maxheight{\ifdim\Gin@nat@height>\textheight\textheight\else\Gin@nat@height\fi}
\makeatother
% Scale images if necessary, so that they will not overflow the page
% margins by default, and it is still possible to overwrite the defaults
% using explicit options in \includegraphics[width, height, ...]{}
\setkeys{Gin}{width=\maxwidth,height=\maxheight,keepaspectratio}
% Set default figure placement to htbp
\makeatletter
\def\fps@figure{htbp}
\makeatother
\setlength{\emergencystretch}{3em} % prevent overfull lines
\providecommand{\tightlist}{%
  \setlength{\itemsep}{0pt}\setlength{\parskip}{0pt}}
\setcounter{secnumdepth}{-\maxdimen} % remove section numbering
\ifluatex
  \usepackage{selnolig}  % disable illegal ligatures
\fi

\title{futR}
\author{Kirstin Holsman}
\date{}

\begin{document}
\maketitle

{
\setcounter{tocdepth}{2}
\tableofcontents
}
Repo maintained by: Kirstin Holsman\\
Alaska Fisheries Science Center\\
NOAA Fisheries, Seattle WA\\
\textbf{\href{mailto:kirstin.holsman@noaa.gov}{\nolinkurl{kirstin.holsman@noaa.gov}}}\\
\emph{Last updated: Apr 18, 2022}

\begin{center}\rule{0.5\linewidth}{0.5pt}\end{center}

\hypertarget{overview}{%
\section{Overview}\label{overview}}

futR() is a generic Rpackage for fitting recruitment models to stock
assessment estimates of spawning stock biomass and recruitment with or
without climate covariates. The recruitment model is based on Template
Model Builder (\texttt{TMB}) and formultations follow Maunder and
Desriso (2011) using a generalized three parameter stock-recruitment
model with environmental covariates (Deriso 1980; Schnute 1985). This
includes Ricker (logistic), Beverton Holt, log-linear, and log-linear
with biomass lagged by year `y-1'. The model can be fit with and with
out random effects on spawning stock biomass (SSB) and recruitment (R)
(i.e., measurement error on SSB and rec) using the methods of Porch and
Lauretta (2016) and with the optional unbiased estimate of sigma (sensu
Ludwid and Walters 1981, Porch and Lauretta 2016). Environmental
covariates are optional but can be included as main effects or as
interactions.

For more information see Holsman et al.~2020 Climate and trophic
controls on groundfish recruitment in Alaska.

\hypertarget{installing-futr}{%
\subsection{Installing futR()}\label{installing-futr}}

The package can be installed from github using the devtools package:

\begin{Shaded}
\begin{Highlighting}[]
\FunctionTok{install.packages}\NormalTok{(}\StringTok{"devtools"}\NormalTok{)}
\end{Highlighting}
\end{Shaded}

The projection package can then be installed to R directly:

\begin{Shaded}
\begin{Highlighting}[]
\NormalTok{devtools}\SpecialCharTok{::}\FunctionTok{install\_github}\NormalTok{(}\StringTok{"kholsman/futR"}\NormalTok{)}
\end{Highlighting}
\end{Shaded}

\hypertarget{setting-up-the-package}{%
\subsection{Setting up the package:}\label{setting-up-the-package}}

The base function for fitting recruitment requires a data.frame of
recruitment and spawning biomass:

\begin{Shaded}
\begin{Highlighting}[]
 \CommentTok{\# rm(list=ls())}
   
  \CommentTok{\#\_\_\_\_\_\_\_\_\_\_\_\_\_\_\_\_\_\_\_\_\_\_\_\_\_\_\_\_\_\_\_\_\_\_\_\_\_\_\_\_\_\_\_}
  \CommentTok{\# 1. Set things up}
  \CommentTok{\#\_\_\_\_\_\_\_\_\_\_\_\_\_\_\_\_\_\_\_\_\_\_\_\_\_\_\_\_\_\_\_\_\_\_\_\_\_\_\_\_\_\_\_}
  
  \CommentTok{\# e.g., }
\NormalTok{  main  }\OtherTok{\textless{}{-}}  \FunctionTok{getwd}\NormalTok{()}
  \FunctionTok{setwd}\NormalTok{(main)}
  
  \CommentTok{\# load data, packages, setup, etc.}
  \FunctionTok{source}\NormalTok{(}\StringTok{"R/make.R"}\NormalTok{)}
    
  \CommentTok{\#\_\_\_\_\_\_\_\_\_\_\_\_\_\_\_\_\_\_\_\_\_\_\_\_\_\_\_\_\_\_\_\_\_\_\_\_\_\_\_\_\_\_\_}
  \CommentTok{\# 2. Compile futR }
  \CommentTok{\#\_\_\_\_\_\_\_\_\_\_\_\_\_\_\_\_\_\_\_\_\_\_\_\_\_\_\_\_\_\_\_\_\_\_\_\_\_\_\_\_\_\_\_}
  
  \FunctionTok{compile}\NormalTok{(}\StringTok{\textquotesingle{}src/futR.cpp\textquotesingle{}}\NormalTok{) }\CommentTok{\# this will generate warnings {-} they can be ignored if "0" is returned}
\end{Highlighting}
\end{Shaded}

\hypertarget{options}{%
\subsection{Options}\label{options}}

Now we can fit a set of models with and without covariates. There are
various switches for fitting models:

\hypertarget{recruitment-formulations-rectype}{%
\subsubsection{Recruitment formulations
(rectype):}\label{recruitment-formulations-rectype}}

\begin{enumerate}
\def\labelenumi{\arabic{enumi}.}
\tightlist
\item
  Linear (gamma = 0)
\item
  Beverton Holt (gamma = -1)
\item
  Ricker (0 \textless{} gamma \textless1 ) ; gamma is estimated
  (tMethod):\\
  \emph{a. }link = cloglog\\
  \emph{b. }link = logit
\item
  Exponential (gamma=1, b\textless0)
\end{enumerate}

\hypertarget{observation-error-options-sigmethod}{%
\subsubsection{Observation error options
(sigMethod):}\label{observation-error-options-sigmethod}}

\begin{enumerate}
\def\labelenumi{\arabic{enumi}.}
\setcounter{enumi}{-1}
\tightlist
\item
  No observation error (tau = 0)
\item
  estimate sigma, random effects on SSB if tau \textgreater0, tau input
\item
  unbiased sigma estimate, tau input
\item
  as in 1 but with defined measurement error for rec (indep of random
  effects on Spawners/SSB)
\item
  as in 1 but with defined measurement error for rec and Spawners/SSB)
\end{enumerate}

\hypertarget{link-options-tmethod}{%
\subsubsection{Link options (tMethod):}\label{link-options-tmethod}}

\begin{enumerate}
\def\labelenumi{\arabic{enumi}.}
\tightlist
\item
  cloglog link (g = 1-exp(-exp(gamma)))
\item
  logit link (g = exp(gamma)/(1+exp(gamma)))
\end{enumerate}

\hypertarget{environmental-effects-if-set-to-true-in-estparm}{%
\subsubsection{Environmental effects (if set to ``TRUE''" in
estparm):}\label{environmental-effects-if-set-to-true-in-estparm}}

\begin{itemize}
\tightlist
\item
  beta = effects on pre-lavarl/ effective number of spawners
\item
  lamba = effects on post-spawning success (e.g., age 0+ survival)
\end{itemize}

\hypertarget{base-model}{%
\section{Base model}\label{base-model}}

Let's start by fitting based models (no climate covariates) with
different options for observation error.

\hypertarget{observation-error-options-sigmethod-1}{%
\subsubsection{Observation error options
(sigMethod):}\label{observation-error-options-sigmethod-1}}

\begin{enumerate}
\def\labelenumi{\arabic{enumi}.}
\setcounter{enumi}{-1}
\tightlist
\item
  No observation error (tau = 0)
\item
  estimate sigma, random effects on SSB if tau \textgreater0, tau input
\item
  unbiased sigma estimate, tau input
\item
  as in 1 but with defined measurement error for rec (indep of random
  effects on Spawners/SSB)
\item
  as in 1 but with defined measurement error for rec and Spawners/SSB)
\end{enumerate}

\hypertarget{no-observation-error-tau-0}{%
\subsection{0. No observation error (tau =
0)}\label{no-observation-error-tau-0}}

\begin{Shaded}
\begin{Highlighting}[]
  \CommentTok{\# set up some demo data:}
\NormalTok{  rec        }\OtherTok{\textless{}{-}}\NormalTok{  rec\_dat[[}\DecValTok{1}\NormalTok{]]}
\NormalTok{  env        }\OtherTok{\textless{}{-}}\NormalTok{  env\_covars}
  
  \CommentTok{\# z score the covariates:}
\NormalTok{  env[}\DecValTok{1}\NormalTok{,]    }\OtherTok{\textless{}{-}}  \FunctionTok{as.numeric}\NormalTok{(}\FunctionTok{scale}\NormalTok{(env\_covars[}\DecValTok{1}\NormalTok{,]))}
\NormalTok{  env[}\DecValTok{2}\NormalTok{,]    }\OtherTok{\textless{}{-}}  \FunctionTok{as.numeric}\NormalTok{(}\FunctionTok{scale}\NormalTok{(env\_covars[}\DecValTok{2}\NormalTok{,]))}
\NormalTok{  ration     }\OtherTok{\textless{}{-}}\NormalTok{  ration\_tmb[,}\DecValTok{1}\NormalTok{]}
  
  
  \CommentTok{\# 3.2 Set up data}
\NormalTok{  PAR}\SpecialCharTok{$}\NormalTok{phases}
\NormalTok{  PAR}\SpecialCharTok{$}\NormalTok{estparams}
  
  \CommentTok{\# which parameters to estimate with futR?}
\NormalTok{  phases  }\OtherTok{=} \FunctionTok{c}\NormalTok{(}
    \AttributeTok{log\_a        =} \DecValTok{1}\NormalTok{, }
    \AttributeTok{log\_b        =} \DecValTok{1}\NormalTok{, }
    \CommentTok{\#logit\_tau     = TRUE,}
    \AttributeTok{beta         =} \DecValTok{1}\NormalTok{,}
    \AttributeTok{lambda       =} \DecValTok{1}\NormalTok{,}
    \AttributeTok{epsi\_s       =} \DecValTok{1}\NormalTok{,}
    \AttributeTok{logsigma     =} \DecValTok{1}\NormalTok{)}
  
\NormalTok{   estparams  }\OtherTok{=} \FunctionTok{c}\NormalTok{(}
    \AttributeTok{log\_a        =} \ConstantTok{TRUE}\NormalTok{, }
    \AttributeTok{log\_b        =} \ConstantTok{TRUE}\NormalTok{, }
    \CommentTok{\#logit\_tau     = TRUE,}
    \AttributeTok{beta         =} \ConstantTok{FALSE}\NormalTok{,}
    \AttributeTok{lambda       =} \ConstantTok{TRUE}\NormalTok{,}
    \AttributeTok{epsi\_s       =} \ConstantTok{FALSE}\NormalTok{,}
    \AttributeTok{logsigma     =} \ConstantTok{TRUE}\NormalTok{)}
  
  \CommentTok{\# rec\_noerr        \textless{}{-} rec}
  \CommentTok{\# rec\_noerr$sdRobs \textless{}{-} 0}
  
  \CommentTok{\# makeDat will make the input values, data, and phases for the model:}
\NormalTok{  datlist  }\OtherTok{\textless{}{-}}  \FunctionTok{makeDat}\NormalTok{(}
                    \AttributeTok{tauIN      =}  \DecValTok{1}\NormalTok{,}
                    \AttributeTok{sigMethod  =}  \DecValTok{1}\NormalTok{, }\CommentTok{\#estimate sigma, random effects on SSB if tau \textgreater{}0, tau input}
                    \AttributeTok{tMethod    =}  \DecValTok{1}\NormalTok{,}
                    \AttributeTok{estparams  =}\NormalTok{  estparams,}
                    \AttributeTok{typeIN     =}  \DecValTok{4}\NormalTok{,}
                    \AttributeTok{rec\_years  =}\NormalTok{  rec}\SpecialCharTok{$}\NormalTok{years,}
                    \AttributeTok{Rec        =}\NormalTok{  rec}\SpecialCharTok{$}\NormalTok{Robs,}
                    \AttributeTok{SSB        =}\NormalTok{  rec}\SpecialCharTok{$}\NormalTok{SSB,}
                    \AttributeTok{sdSSB      =}\NormalTok{  rec}\SpecialCharTok{$}\NormalTok{sdSSB,}
                    \AttributeTok{sdRec      =}\NormalTok{  rec}\SpecialCharTok{$}\NormalTok{sdRobs,}
                    \AttributeTok{covars     =}  \ConstantTok{NULL}\NormalTok{,}
                    \AttributeTok{covars\_sd  =}  \ConstantTok{NULL}\NormalTok{)}
          
    \CommentTok{\# run the basic model}
\NormalTok{  wd0 }\OtherTok{\textless{}{-}} \FunctionTok{getwd}\NormalTok{()}
  \FunctionTok{setwd}\NormalTok{(}\StringTok{"../src"}\NormalTok{)}
\NormalTok{  mm1 }\OtherTok{\textless{}{-}}  \FunctionTok{runmod}\NormalTok{(}\AttributeTok{dlistIN=}\NormalTok{datlist,}\AttributeTok{version=}\StringTok{\textquotesingle{}futR\textquotesingle{}}\NormalTok{,}\AttributeTok{recompile=}\NormalTok{T,}\AttributeTok{simulate=}\ConstantTok{TRUE}\NormalTok{,}\AttributeTok{simnitr =} \DecValTok{1000}\NormalTok{)}
  
\NormalTok{  df1 }\OtherTok{\textless{}{-}}  \FunctionTok{data.frame}\NormalTok{(}\AttributeTok{model =} \StringTok{"mm1"}\NormalTok{, }
                     \AttributeTok{estimate=}\FunctionTok{as.vector}\NormalTok{(mm1}\SpecialCharTok{$}\NormalTok{sim), }
                     \AttributeTok{parameter=}\FunctionTok{names}\NormalTok{( mm1}\SpecialCharTok{$}\NormalTok{mle)[}\FunctionTok{row}\NormalTok{(mm1}\SpecialCharTok{$}\NormalTok{sim)])}
  
\NormalTok{  mu   }\OtherTok{\textless{}{-}}\NormalTok{ df1}\SpecialCharTok{\%\textgreater{}\%}\FunctionTok{group\_by}\NormalTok{(model,parameter)}\SpecialCharTok{\%\textgreater{}\%}\FunctionTok{summarise}\NormalTok{(}\AttributeTok{grp.mean=}\FunctionTok{mean}\NormalTok{(estimate))}
\NormalTok{  peak }\OtherTok{\textless{}{-}}\NormalTok{ df1}\SpecialCharTok{\%\textgreater{}\%}\FunctionTok{group\_by}\NormalTok{(model,parameter)}\SpecialCharTok{\%\textgreater{}\%}
    \FunctionTok{count}\NormalTok{(parameter,}\FunctionTok{round}\NormalTok{(estimate,}\DecValTok{1}\NormalTok{))}\SpecialCharTok{\%\textgreater{}\%}
    \FunctionTok{slice}\NormalTok{(}\FunctionTok{which.max}\NormalTok{(n))}
  \FunctionTok{names}\NormalTok{(peak)}\OtherTok{\textless{}{-}} \FunctionTok{c}\NormalTok{(}\StringTok{"model"}\NormalTok{,}\StringTok{"parameter"}\NormalTok{,}\StringTok{"freq"}\NormalTok{,}\StringTok{"n"}\NormalTok{)}
  \CommentTok{\# now plot the denisty of each parm:}
\NormalTok{  p }\OtherTok{\textless{}{-}}
    \FunctionTok{ggplot}\NormalTok{(}\AttributeTok{data=}\NormalTok{df1) }\SpecialCharTok{+}
    \FunctionTok{geom\_density}\NormalTok{( }\FunctionTok{aes}\NormalTok{(}\AttributeTok{x=}\NormalTok{estimate, }\AttributeTok{color=}\NormalTok{parameter))}\SpecialCharTok{+}
    \FunctionTok{facet\_wrap}\NormalTok{(}\SpecialCharTok{\textasciitilde{}}\NormalTok{parameter,}\AttributeTok{scales=}\StringTok{"free"}\NormalTok{)}\SpecialCharTok{+} 
    \FunctionTok{geom\_vline}\NormalTok{(}\AttributeTok{data=}\NormalTok{peak,}\FunctionTok{aes}\NormalTok{(}\AttributeTok{xintercept=}\NormalTok{freq),}
              \AttributeTok{color=}\StringTok{"blue"}\NormalTok{, }\AttributeTok{linetype=}\StringTok{"dashed"}\NormalTok{, }\AttributeTok{size=}\DecValTok{1}\NormalTok{)}\SpecialCharTok{+}
    \FunctionTok{theme\_kir\_EBM}\NormalTok{()}
\NormalTok{  p}
\end{Highlighting}
\end{Shaded}

\hypertarget{est.-sigma-rand-effects-on-ssb-if-tau-0}{%
\subsection{1. est. sigma, rand effects on SSB if tau \textgreater{}
0}\label{est.-sigma-rand-effects-on-ssb-if-tau-0}}

\begin{Shaded}
\begin{Highlighting}[]
\NormalTok{  datlist}\SpecialCharTok{$}\NormalTok{rs\_dat}\SpecialCharTok{$}\NormalTok{tau  }\OtherTok{\textless{}{-}}  \FloatTok{0.000001}  
  \CommentTok{\# recall that sigMethod == 1 when creating datlist: i.e., }
  \CommentTok{\# estimate sigma, random effects on SSB if tau \textgreater{}0, tau input}
    
  \CommentTok{\# re{-}run the model with tau }
\NormalTok{  mm1\_t0 }\OtherTok{\textless{}{-}} \FunctionTok{runmod}\NormalTok{(}\AttributeTok{dlistIN=}\NormalTok{datlist,}\AttributeTok{version=}\StringTok{"futR"}\NormalTok{,}\AttributeTok{recompile=}\NormalTok{F,}\AttributeTok{simulate=}\ConstantTok{TRUE}\NormalTok{)}
\NormalTok{  df1\_t0 }\OtherTok{\textless{}{-}} \FunctionTok{data.frame}\NormalTok{(}
    \AttributeTok{estimate =} \FunctionTok{as.vector}\NormalTok{(mm1\_t0}\SpecialCharTok{$}\NormalTok{sim),}
    \AttributeTok{parameter =} \FunctionTok{names}\NormalTok{( mm1\_t0}\SpecialCharTok{$}\NormalTok{mle)[}\FunctionTok{row}\NormalTok{(mm1\_t0}\SpecialCharTok{$}\NormalTok{sim)])}
   
  
   
\NormalTok{  df1\_t0 }\OtherTok{\textless{}{-}}  \FunctionTok{data.frame}\NormalTok{(}\AttributeTok{model =} \StringTok{"mm1\_t0"}\NormalTok{, }
                     \AttributeTok{estimate=}\FunctionTok{as.vector}\NormalTok{(mm1\_t0}\SpecialCharTok{$}\NormalTok{sim), }
                     \AttributeTok{parameter=}\FunctionTok{names}\NormalTok{( mm1\_t0}\SpecialCharTok{$}\NormalTok{mle)[}\FunctionTok{row}\NormalTok{(mm1\_t0}\SpecialCharTok{$}\NormalTok{sim)])}
\NormalTok{  df }\OtherTok{\textless{}{-}} \FunctionTok{rbind}\NormalTok{(df1, df1\_t0)}
\NormalTok{  mu   }\OtherTok{\textless{}{-}}\NormalTok{ df1}\SpecialCharTok{\%\textgreater{}\%}\FunctionTok{group\_by}\NormalTok{(model,parameter)}\SpecialCharTok{\%\textgreater{}\%}\FunctionTok{summarise}\NormalTok{(}\AttributeTok{grp.mean=}\FunctionTok{mean}\NormalTok{(estimate))}
\NormalTok{  peak }\OtherTok{\textless{}{-}}\NormalTok{ df1}\SpecialCharTok{\%\textgreater{}\%}\FunctionTok{group\_by}\NormalTok{(model,parameter)}\SpecialCharTok{\%\textgreater{}\%}
    \FunctionTok{count}\NormalTok{(parameter,}\FunctionTok{round}\NormalTok{(estimate,}\DecValTok{1}\NormalTok{))}\SpecialCharTok{\%\textgreater{}\%}
    \FunctionTok{slice}\NormalTok{(}\FunctionTok{which.max}\NormalTok{(n))}
  \FunctionTok{names}\NormalTok{(peak)}\OtherTok{\textless{}{-}} \FunctionTok{c}\NormalTok{(}\StringTok{"model"}\NormalTok{,}\StringTok{"parameter"}\NormalTok{,}\StringTok{"freq"}\NormalTok{,}\StringTok{"n"}\NormalTok{)}
  \CommentTok{\# now plot the denisty of each parm:}
\NormalTok{  p }\OtherTok{\textless{}{-}}
    \FunctionTok{ggplot}\NormalTok{(}\AttributeTok{data=}\NormalTok{df) }\SpecialCharTok{+}
    \FunctionTok{geom\_density}\NormalTok{( }\FunctionTok{aes}\NormalTok{(}\AttributeTok{x=}\NormalTok{estimate, }\AttributeTok{color=}\NormalTok{model))}\SpecialCharTok{+}
    \FunctionTok{facet\_wrap}\NormalTok{(}\SpecialCharTok{\textasciitilde{}}\NormalTok{parameter,}\AttributeTok{scales=}\StringTok{"free"}\NormalTok{)}\SpecialCharTok{+} 
    \FunctionTok{geom\_vline}\NormalTok{(}\AttributeTok{data=}\NormalTok{peak,}\FunctionTok{aes}\NormalTok{(}\AttributeTok{xintercept=}\NormalTok{freq, }\AttributeTok{color =}\NormalTok{ model), }\AttributeTok{linetype=}\StringTok{"dashed"}\NormalTok{, }\AttributeTok{size=}\DecValTok{1}\NormalTok{)}\SpecialCharTok{+}
    \FunctionTok{theme\_kir\_EBM}\NormalTok{()}
\NormalTok{  p  }
\end{Highlighting}
\end{Shaded}

\hypertarget{unbiased-sigma-estimate-tau-input}{%
\subsection{2. unbiased sigma estimate, tau
input}\label{unbiased-sigma-estimate-tau-input}}

\begin{Shaded}
\begin{Highlighting}[]
\NormalTok{  datlist}\SpecialCharTok{$}\NormalTok{rs\_dat}\SpecialCharTok{$}\NormalTok{tau  }\OtherTok{\textless{}{-}}  \FloatTok{0.000001}  
  \CommentTok{\# recall that sigMethod == 1 when creating datlist: i.e., }
  \CommentTok{\# estimate sigma, random effects on SSB if tau \textgreater{}0, tau input}
    
  \CommentTok{\# re{-}run the model with tau }
\NormalTok{  mm1\_t0 }\OtherTok{\textless{}{-}} \FunctionTok{runmod}\NormalTok{(}\AttributeTok{dlistIN=}\NormalTok{datlist,}\AttributeTok{version=}\StringTok{"futR"}\NormalTok{,}\AttributeTok{recompile=}\NormalTok{F,}\AttributeTok{simulate=}\ConstantTok{TRUE}\NormalTok{)}
\NormalTok{  df1\_t0 }\OtherTok{\textless{}{-}} \FunctionTok{data.frame}\NormalTok{(}
    \AttributeTok{estimate =} \FunctionTok{as.vector}\NormalTok{(mm1\_t0}\SpecialCharTok{$}\NormalTok{sim),}
    \AttributeTok{parameter =} \FunctionTok{names}\NormalTok{( mm1\_t0}\SpecialCharTok{$}\NormalTok{mle)[}\FunctionTok{row}\NormalTok{(mm1\_t0}\SpecialCharTok{$}\NormalTok{sim)])}
   
  
   
\NormalTok{  df1\_t0 }\OtherTok{\textless{}{-}}  \FunctionTok{data.frame}\NormalTok{(}\AttributeTok{model =} \StringTok{"mm1\_t0"}\NormalTok{, }
                     \AttributeTok{estimate=}\FunctionTok{as.vector}\NormalTok{(mm1\_t0}\SpecialCharTok{$}\NormalTok{sim), }
                     \AttributeTok{parameter=}\FunctionTok{names}\NormalTok{( mm1\_t0}\SpecialCharTok{$}\NormalTok{mle)[}\FunctionTok{row}\NormalTok{(mm1\_t0}\SpecialCharTok{$}\NormalTok{sim)])}
\NormalTok{  df }\OtherTok{\textless{}{-}} \FunctionTok{rbind}\NormalTok{(df1, df1\_t0)}
\NormalTok{  mu   }\OtherTok{\textless{}{-}}\NormalTok{ df1}\SpecialCharTok{\%\textgreater{}\%}\FunctionTok{group\_by}\NormalTok{(model,parameter)}\SpecialCharTok{\%\textgreater{}\%}\FunctionTok{summarise}\NormalTok{(}\AttributeTok{grp.mean=}\FunctionTok{mean}\NormalTok{(estimate))}
\NormalTok{  peak }\OtherTok{\textless{}{-}}\NormalTok{ df1}\SpecialCharTok{\%\textgreater{}\%}\FunctionTok{group\_by}\NormalTok{(model,parameter)}\SpecialCharTok{\%\textgreater{}\%}
    \FunctionTok{count}\NormalTok{(parameter,}\FunctionTok{round}\NormalTok{(estimate,}\DecValTok{1}\NormalTok{))}\SpecialCharTok{\%\textgreater{}\%}
    \FunctionTok{slice}\NormalTok{(}\FunctionTok{which.max}\NormalTok{(n))}
  \FunctionTok{names}\NormalTok{(peak)}\OtherTok{\textless{}{-}} \FunctionTok{c}\NormalTok{(}\StringTok{"model"}\NormalTok{,}\StringTok{"parameter"}\NormalTok{,}\StringTok{"freq"}\NormalTok{,}\StringTok{"n"}\NormalTok{)}
  \CommentTok{\# now plot the denisty of each parm:}
\NormalTok{  p }\OtherTok{\textless{}{-}}
    \FunctionTok{ggplot}\NormalTok{(}\AttributeTok{data=}\NormalTok{df) }\SpecialCharTok{+}
    \FunctionTok{geom\_density}\NormalTok{( }\FunctionTok{aes}\NormalTok{(}\AttributeTok{x=}\NormalTok{estimate, }\AttributeTok{color=}\NormalTok{model))}\SpecialCharTok{+}
    \FunctionTok{facet\_wrap}\NormalTok{(}\SpecialCharTok{\textasciitilde{}}\NormalTok{parameter,}\AttributeTok{scales=}\StringTok{"free"}\NormalTok{)}\SpecialCharTok{+} 
    \FunctionTok{geom\_vline}\NormalTok{(}\AttributeTok{data=}\NormalTok{peak,}\FunctionTok{aes}\NormalTok{(}\AttributeTok{xintercept=}\NormalTok{freq, }\AttributeTok{color =}\NormalTok{ model), }\AttributeTok{linetype=}\StringTok{"dashed"}\NormalTok{, }\AttributeTok{size=}\DecValTok{1}\NormalTok{)}\SpecialCharTok{+}
    \FunctionTok{theme\_kir\_EBM}\NormalTok{()}
\NormalTok{  p  }
\end{Highlighting}
\end{Shaded}


\end{document}
